
%%%%%%%%%%%%%%%%%%%%%%%%%%%%%%%%%%%%%%%%%%%%%%%%%%%%%%%%%%%%%%%%%%%%%%%%%%%%%%%%%%%%%%%%%%%%%%%%%%%%%%%%%%%%%%%%%%%%%%%%
%%
%% 情報通信システム工学科 卒業論文テンプレート
%%
%%%%%%%%%%%%%%%%%%%%%%%%%%%%%%%%%%%%%%%%%%%%%%%%%%%%%%%%%%%%%%%%%%%%%%%%%%%%%%%%%%%%%%%%%%%%%%%%%%%%%%%%%%%%%%%%%%%%%%%%
\documentclass[11pt]{icsthesis}
\usepackage[dvipdfmx]{graphicx}
\usepackage[fleqn]{amsmath}
\usepackage{fancyhdr}
\usepackage{bm}
\usepackage{bigstrut}
\usepackage{multirow}
\usepackage{amsfonts}

%%----------------------------------------------------------------------------------------------------------------------
%% 表紙の記載事項
%%----------------------------------------------------------------------------------------------------------------------
%% ToDo: 論文タイトル
\title{卒業論文テンプレート}
%% ToDo: 著者一覧
\authorA{21A5041}{記内 勇太}
\authorB{21A5101}{古山 孔亮}
%\authorC{21A5003}{氏 名3}
%\authorD{21A5004}{氏 名4}

%% ToDo: 指導教員
\supervisor{菅原 真司 教授}

%% ToDo: 提出日
\date{令和7年1月30日}

%%----------------------------------------------------------------------------------------------------------------------
\begin{document}
\maketitle
%-----------------------------------------------------------
% ヘッダ・フッタ設定
\pagestyle{fancy}
\fancyhead[R]{\nouppercase{\fontsize{10.5pt}{0pt}\selectfont\rightmark}}
\fancyhead[L]{\nouppercase{\fontsize{10.5pt}{0pt}\selectfont\leftmark}}
%\fancyhead[R]{\nouppercase{\small\rightmark}}
%\fancyhead[L]{\nouppercase{\small\leftmark}}
\fancyfoot[C]{--\ \thepage\ --}
\renewcommand{\headrulewidth}{0.3truemm}
%-----------------------------------------------------------
%% 目次
\pagenumbering{roman}
\setcounter{tocdepth}{4}
\pagestyle{fancy}
\fancyfoot[C]{--\ \thepage\ --}
{\makeatletter
\let\ps@jpl@in\ps@empty
\makeatother
\pagestyle{plain}
\tableofcontents
\clearpage}
\pagenumbering{arabic}
%%======================================================================================================================
%% ここから本文 ::::::::::::::::::::::::::::::::::::::::::::::::::::::::::::::::::::::::::::::::::::::::::::::::::::::::
%%======================================================================================================================

%%=============第1章=============
\chapter{まえがき}
本書は,情報通信システム工学科卒業論文用\LaTeX テンプレートファイルの説明書である.
\LaTeX (必要に応じて,\BibTeX)はそれなりに使えることが前提である.

\section{研究背景}
近年,インターネットは,情報交換・共有システムとして,社会・経済のインフラストラクチャの役割を担っている.しかし,全ての地域に高度な通信サービスを提供することは,コストが掛かる上に周波数の有限性から現実的ではない.
また,ディジタルデバイド地域のインフラ構築や災害時などで本来の通信インフラの機能や性能が低下・停止した場合を考え,場所・時間を問わず利用できる情報サービスの実現が必要とされている.

そのため,従来では想定されていなかった環境を含むエンドツーエンドの情報伝達において,従来のTCP/IP(Transmission Control Protocol/Internet Protocol)
技術を拡張し,中継ノード及びエンドノードの機能を再設計する必要がある.昔から車や人などが物理的に情報を運ぶやり方は
存在していたが,それらをTCP/IP技術の問題認識から発生した技術領域としてインターネット通信の枠組みで考えられたのが,DTNである.

TCP/IP技術におけるDTNの概念はビントン・サーフ氏によって提案され,惑星間インターネットの構築が基になっている.その惑星間インターネットには2つの大きな問題点が挙げられる.まず,惑星間というのは光速で進む電波であっても,届くのに長い時間を要する.
次に,通信機が惑星によって見えなくなり,その間,通信が途絶える.これらの問題をそれぞれ,「通信遅延」,「通信途絶」として研究されてきた.

DTNの特徴的な技術として,蓄積運搬形転送がある.これは,再開を待つ,あるいは最適なタイミングを待つための一時的「蓄積」と,物理的に車や列車を用いて情報を運ぶ「運搬」を用いる必要がある.蓄積運搬形転送の中で,疎密度モバイルアドホック
網における感染形中継転送方式(Epidemic Routing)が確率的中継転送方式の最も代表的な方式として存在する.Epidemic Routingでは,メッセージを保持する端末が他の端末と通信可能となった際,常に保持しているメッセージの複製を
他の端末に送信する.結果的に伝染病のように広がり,いずれかのメッセージが宛先端末に到達することになる.しかし,Epidemic Routingではメッセージの到達を引き換えに,端末のバッファや網内に多くの複製メッセージが残存してしまう.その
ため,回復手法(Recovery Scheme)や除去パケット(Anti Packet)を使用したルーティング手法などが提案されてきた[1].

本研究では,DTN環境において「メッセージ到達率」,「メッセージ平均待ち時間」,「メッセージ削除数」に着目して,ネットワーク負荷の削減・効率的なメッセージ転送の実現を目指す.

\section{DTN}
DTNとは,Delay,Disruption,Disconnection Tolerant Networkingの略で,大きな遅延や通信途絶,繋がっていない状態の方が多い環境においても,適切な情報通信を現実的なコストで実現できる技術である.要素技術として,バンドルプロトコルとストアアンド
フォワード方式が存在する.インターネットでのデータ転送にはトランスポート層のTCPを使用するが,通信に遅延や切断が生じると,通信品質が劣化してしまう.そのため,TCPによる通信では,エンドツーエンドのリンクが安定している必要がある.DTNでは,
不安定なリンクにも対応しなければならないため,トランスポート層の上に「バンドル」と呼ばれるプロトコルを乗せることで劣通信環境に対応する.バンドルプロトコルの中でも,データ転送にはストアアンドフォワード方式を用いる.ストアアンド
フォワード方式は,通信が不可能な状態である時にデータを蓄積し,通信が可能な状態である時にデータを転送する方式である.

%%=============第2章=============
\chapter{DTNにおける従来手法とその問題}
\section{Epidemic Routing}
蓄積運搬形転送に基づく最も初期に提案された手法に,感染形中継転送方式であるEpidemic Routingと呼ばれる手法が存在する.メッセージを保持する端末が他の端末と接触をすることで通信を行い,複製メッセージを拡散させながら網内の末端までメッセージを
届けることを可能にし,いずれかのメッセージが宛先端末に到達する.この手法は,網資源がある場合には,あらゆる方式の中で最も優れた遅延性能を示す.しかし,受信端末は複製メッセージをさらに複製するため,網資源を最も消費する方式
でもある.


\section{ワクチン回復手法}

%%=============第3章=============
\chapter{提案手法}
\section{Priority Message Router}
Priority Message Router(以下,PMRouter)とは,端末がメッセージに優先度を設け,他ノードと接触したらそのノードのすれ違ったノード履歴より,自身の保有するメッセージの宛先と同一のメッセージを探す.存在した時に,そのメッセージの
優先度を向上させ,バッファ内に存在できる時間をコントロールする.優先度の向上したメッセージに対しTTL(メッセージ有効時間)を変更することで,「メッセージ到達率」,「メッセージ平均待ち時間」,「メッセージ削除数」の
改善を図る手法である.基本的なメッセージの転送は感染形中継転送方式であり,優先度を設けることから,Epidemic RoutingとMax Propを拡張したものとなる.

\section{PMRouterの構成とアルゴリズム}
PMRouterでは,始めにEpidemic Routingの環境を用意する.そこで,各メッセージに優先度を設けるためのメソッドを追加する(ここまではMax Propと同様).次に各ノードのすれ違ったノードを格納するためのリストを追加し,自身の保有するメッセージの宛先と
比較する.最後に,自身の保有するメッセージの宛先が他ノードのすれ違ったノードのリストに存在していれば,その宛先メッセージの優先度を向上させ,TTLをコントールする.

%%=============第4章=============
\chapter{評価}
\section{評価方法}
計算機シミュレーションにより,「メッセージ到達率」,「メッセージ平均待ち時間」,「メッセージ削除数」を評価尺度として提案方式の性能評価を検証する.また,従来手法(Epidemic Routing及び回復手法)と提案手法の性能評価を比較し,提案手法の
実用性を検証する.

\section{実験環境}
本研究では,The ONEシミュレータ(The Opportunistic Network Environment Simulator)用いて実験を行う.The ONEシミュレータは,DTNの評価実験をするために作成されたフリーのネットワークシミュレータである.任意のルーティングや
モビリティパターンを導入することができる.時間の遷移方式はタイムステップ型であり,標準では0.1秒毎に遷移する.また,通信可能距離と通信速度のみを考慮し,パケットのロスト,建物,電波干渉等による通信範囲の削減,保有メッセージの把握等の細かい
部分は考慮していない.このシミュレータではバンドル層以上のみでシミュレーションを行っているため,トランスポート層以下に関わる電波の輻輳や衝突等については,パラメータで妥当な値を設定しなければならない.

\section{条件とシミュレーションパラメータ}
本研究を行う上での条件を以下に示す.
・デフォルトマップ(初期状態のThe ONEシミュレータに搭載されている)を用いる.
・
・
・
本研究を行うためのシミュレーションパラメータを表?に示す.

%%=============第4章=============
\chapter{結論}



%%======================================================================================================================
%% ここまで本文 ::::::::::::::::::::::::::::::::::::::::::::::::::::::::::::::::::::::::::::::::::::::::::::::::::::::::
%%======================================================================================================================

%% 謝辞
\clearpage
\fancyhead[L]{}\fancyhead[R]{}
\renewcommand{\headrulewidth}{0truemm}
\section*{謝辞}
本研究を遂行し,卒業論文をまとめるにあたって,ご指導ならびにご助言して頂き,研究室活動全般にわたってお世話になりました菅原真司教授に感謝致します.

%%----------------------------------------------------------------------------------------------------------------------
%% 参考文献
\clearpage
%\nocite{*}
\bibliographystyle{icsthesis}
\fancyhead[L]{\nouppercase{\small\leftmark}}\fancyhead[R]{}
\renewcommand{\headrulewidth}{0.3truemm}
\bibliography{thesis}

%%----------------------------------------------------------------------------------------------------------------------
%% 付録: 不要なら,最後の \end{document} を残して,これ以降の行を消す.
%%
\clearpage
\fancyhead[L]{\nouppercase{\small\leftmark}}
\fancyhead[R]{\nouppercase{\small\rightmark}}
\fancyfoot[C]{--\ \thepage\ --}
\renewcommand{\headrulewidth}{0.3truemm}
\appendix
\chapter{序論}
ここから付録のページである.章や節の番号付けが変わるだけで本文の章や節と同じように記述できる.
不要な場合は,``\textsf{thesis.tex}''内の指示に従い,不要な行を削除する.

ここは「付録章」である.\textsf{\yen appendix}以降に,\textsf{\yen chapter\{見出し文字列\}}と書けば,
「付録章見出し」が設定される.
「章見出し」と同じ文字サイズ,同じ上下余白である.

\section{付録節見出し}
ここは,「付録節」である.\textsf{\yen appendix}以降に\textsf{\yen section\{見出し文字列\}}と書けば,
「付録節見出し」が設定される.
「節見出し」と同じ文字サイズ,同じ上下余白である.

\subsection{付録小節見出し}
ここは,「付録小節」である.\textsf{\yen appendix}以降に\textsf{\yen subsection\{見出し文字列\}}と書けば,
「付録小節見出し」が設定される.
「小節見出し」と同じ文字サイズ,同じ上下余白である.

\subsubsection{付録小小見出し}
ここは,「付録小小節」である.\textsf{\yen appendix}以降に\textsf{\yen subsubsection\{見出し文字列\}}と書けば,
「付録小小節見出し」が設定される.
「小小節見出し」と同じ文字サイズ,同じ上下余白である.


\end{document}
