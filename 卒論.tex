%%%%%%%%%%%%%%%%%%%%%%%%%%%%%%%%%%%%%%%%%%%%%%%%%%%%%%%%%%%%%%%%%%%%%%%%%%%%%%%%%%%%%%%%%%%%%%%%%%%%%%%%%%%%%%%%%%%%%%%%
%%
%% 情報通信システム工学科 卒業論文テンプレート
%%
%%%%%%%%%%%%%%%%%%%%%%%%%%%%%%%%%%%%%%%%%%%%%%%%%%%%%%%%%%%%%%%%%%%%%%%%%%%%%%%%%%%%%%%%%%%%%%%%%%%%%%%%%%%%%%%%%%%%%%%%
\documentclass[11pt]{icsthesis}
\usepackage[dvipdfmx]{graphicx}
\usepackage[fleqn]{amsmath}
\usepackage{fancyhdr}
\usepackage{bm}
\usepackage{bigstrut}
\usepackage{multirow}
\usepackage{amsfonts}

%%----------------------------------------------------------------------------------------------------------------------
%% 表紙の記載事項
%%----------------------------------------------------------------------------------------------------------------------
%% ToDo: 論文タイトル
\title{卒業論文テンプレート}
%% ToDo: 著者一覧
\authorA{21A5041}{記内 勇太}
\authorB{21A5101}{古山 孔亮}
%\authorC{21A5003}{氏 名3}
%\authorD{21A5004}{氏 名4}

%% ToDo: 指導教員
\supervisor{菅原 真司 教授}

%% ToDo: 提出日
\date{令和7年1月30日}

%%----------------------------------------------------------------------------------------------------------------------
\begin{document}
\maketitle
%-----------------------------------------------------------
% ヘッダ・フッタ設定
\pagestyle{fancy}
\fancyhead[R]{\nouppercase{\fontsize{10.5pt}{0pt}\selectfont\rightmark}}
\fancyhead[L]{\nouppercase{\fontsize{10.5pt}{0pt}\selectfont\leftmark}}
%\fancyhead[R]{\nouppercase{\small\rightmark}}
%\fancyhead[L]{\nouppercase{\small\leftmark}}
\fancyfoot[C]{--\ \thepage\ --}
\renewcommand{\headrulewidth}{0.3truemm}
%-----------------------------------------------------------
%% 目次
\pagenumbering{roman}
\setcounter{tocdepth}{4}
\pagestyle{fancy}
\fancyfoot[C]{--\ \thepage\ --}
{\makeatletter
\let\ps@jpl@in\ps@empty
\makeatother
\pagestyle{plain}
\tableofcontents
\clearpage}
\pagenumbering{arabic}
%%======================================================================================================================
%% ここから本文 ::::::::::::::::::::::::::::::::::::::::::::::::::::::::::::::::::::::::::::::::::::::::::::::::::::::::
%%======================================================================================================================

\chapter{まえがき}
本書は,情報通信システム工学科卒業論文用\LaTeX テンプレートファイルの説明書である.
\LaTeX (必要に応じて,\BibTeX)はそれなりに使えることが前提である.

\chapter{研究背景}
近年,インターネットは,情報交換・共有システムとして,社会・経済のインフラストラクチャの役割を担っている.
しかし,全ての地域に高度な通信サービスを提供することは,コストが掛かる上に周波数の有限性から現実的ではない.
また,ディジタルデバイド地域のインフラ構築や災害時などで本来の通信インフラの機能や性能が低下・停止した場合
を考え,場所・時間を問わず利用できる情報サービスの実現が必要とされている.

そこで,従来では想定されていなかった環境を含むエンドツーエンドの情報伝達において,従来のTCP/IP(Transmission Control Protocol/Internet Protocol)
技術を拡張し,中継ノード及びエンドノードの機能を再設計する必要がある.そこで,車や人などが物理的に情報を運ぶやり方は
昔から存在していたが,それらをTCP/IP技術の問題認識から発生した技術領域がとしてインターネット通信の枠組みで考えられたのが,遅延耐性ネットワーク(DTN:Delay,Disruption,Disconnection Tolerant Networking)である.

DTNの概念はビントン・サーフ氏によって提案され,惑星間インターネットの構築が基になっている.その惑星間インターネットには2つの大きな問題点が挙げられる.まず,惑星間というのは光速で進む電波であっても,届くのに長い時間を要する.
次に,通信機が惑星によって見えなくなり,その間,通信が途絶える.これらの問題をそれぞれ,「通信遅延」,「通信途絶」として研究されてきた.

DTNの特徴的な技術として,蓄積運搬形転送がある.

\textsf{\yen chapter\{見出し文字列\}}と書けば,「章見出し」が設定される.

ちなみに,本文の文字サイズは11ptである.

\section{節見出し}
ここは,「節」である.見出しはゴシック14ptである.見出し文字サイズで,上に1行,下に0.5行分程度の余白がある.

\textsf{\yen section\{見出し文字列\}}と書けば,「節見出し」が設定される.

\subsection{小節見出し}
ここは,「小節」である.見出しはゴシック12ptである.で,以下同文...

\textsf{\yen subsection\{見出し文字列\}}と書けば,「小節見出し」が設定される.

\subsubsection{小小節見出し}
ここは,「小小節」である.見出しはゴシック11ptである.で,以下同文...

\textsf{\yen subsubsection\{見出し文字列\}}と書けば,「小小節見出し」が設定される.

\section{文字数と行数}
概ね,1ページ40行,1行45文字となるように設定してある.


%%======================================================================================================================
%% ここまで本文 ::::::::::::::::::::::::::::::::::::::::::::::::::::::::::::::::::::::::::::::::::::::::::::::::::::::::
%%======================================================================================================================

%% 謝辞
\clearpage
\fancyhead[L]{}\fancyhead[R]{}
\renewcommand{\headrulewidth}{0truemm}
\section*{謝辞}
本研究を遂行し,卒業論文をまとめるにあたって,ご指導ならびにご助言して頂き,研究室活動全般にわたってお世話になりました菅原真司教授に感謝致します.

%%----------------------------------------------------------------------------------------------------------------------
%% 参考文献
\clearpage
%\nocite{*}
\bibliographystyle{icsthesis}
\fancyhead[L]{\nouppercase{\small\leftmark}}\fancyhead[R]{}
\renewcommand{\headrulewidth}{0.3truemm}
\bibliography{thesis}

%%----------------------------------------------------------------------------------------------------------------------
%% 付録: 不要なら,最後の \end{document} を残して,これ以降の行を消す.
%%
\clearpage
\fancyhead[L]{\nouppercase{\small\leftmark}}
\fancyhead[R]{\nouppercase{\small\rightmark}}
\fancyfoot[C]{--\ \thepage\ --}
\renewcommand{\headrulewidth}{0.3truemm}
\appendix
\chapter{序論}
ここから付録のページである.章や節の番号付けが変わるだけで本文の章や節と同じように記述できる.
不要な場合は,``\textsf{thesis.tex}''内の指示に従い,不要な行を削除する.

ここは「付録章」である.\textsf{\yen appendix}以降に,\textsf{\yen chapter\{見出し文字列\}}と書けば,
「付録章見出し」が設定される.
「章見出し」と同じ文字サイズ,同じ上下余白である.

\section{付録節見出し}
ここは,「付録節」である.\textsf{\yen appendix}以降に\textsf{\yen section\{見出し文字列\}}と書けば,
「付録節見出し」が設定される.
「節見出し」と同じ文字サイズ,同じ上下余白である.

\subsection{付録小節見出し}
ここは,「付録小節」である.\textsf{\yen appendix}以降に\textsf{\yen subsection\{見出し文字列\}}と書けば,
「付録小節見出し」が設定される.
「小節見出し」と同じ文字サイズ,同じ上下余白である.

\subsubsection{付録小小見出し}
ここは,「付録小小節」である.\textsf{\yen appendix}以降に\textsf{\yen subsubsection\{見出し文字列\}}と書けば,
「付録小小節見出し」が設定される.
「小小節見出し」と同じ文字サイズ,同じ上下余白である.


\end{document}
