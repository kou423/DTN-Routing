
%%%%%%%%%%%%%%%%%%%%%%%%%%%%%%%%%%%%%%%%%%%%%%%%%%%%%%%%%%%%%%%%%%%%%%%%%%%%%%%%%%%%%%%%%%%%%%%%%%%%%%%%%%%%%%%%%%%%%%%%
%%
%% 情報通信システム工学科 卒業論文テンプレート
%%
%%%%%%%%%%%%%%%%%%%%%%%%%%%%%%%%%%%%%%%%%%%%%%%%%%%%%%%%%%%%%%%%%%%%%%%%%%%%%%%%%%%%%%%%%%%%%%%%%%%%%%%%%%%%%%%%%%%%%%%%
\documentclass[11pt]{icsthesis}
\usepackage[dvipdfmx]{graphicx}
\usepackage[fleqn]{amsmath}
\usepackage{fancyhdr}
\usepackage{bm}
\usepackage{bigstrut}
\usepackage{multirow}
\usepackage{amsfonts}
\usepackage{float}

\usepackage{listings}
\renewcommand{\lstlistingname}{表}
\usepackage{framed}

%%----------------------------------------------------------------------------------------------------------------------
%% 表紙の記載事項
%%----------------------------------------------------------------------------------------------------------------------
%% ToDo: 論文タイトル
\title{DTNを用いた効率的メッセージ交換に関する研究}
%% ToDo: 著者一覧
\authorA{21A5041}{記内 勇太}
\authorB{21A5101}{古山 孔亮}
%\authorC{21A5003}{氏 名3}
%\authorD{21A5004}{氏 名4}

%% ToDo: 指導教員
\supervisor{菅原 真司 教授}

%% ToDo: 提出日
\date{令和7年1月30日}

%%----------------------------------------------------------------------------------------------------------------------
\begin{document}
\maketitle
%-----------------------------------------------------------
% ヘッダ・フッタ設定
\pagestyle{fancy}
\fancyhead[R]{\nouppercase{\fontsize{10.5pt}{0pt}\selectfont\rightmark}}
\fancyhead[L]{\nouppercase{\fontsize{10.5pt}{0pt}\selectfont\leftmark}}
%\fancyhead[R]{\nouppercase{\small\rightmark}}
%\fancyhead[L]{\nouppercase{\small\leftmark}}
\fancyfoot[C]{--\ \thepage\ --}
\renewcommand{\headrulewidth}{0.3truemm}
%-----------------------------------------------------------
%% 目次
\pagenumbering{roman}
\setcounter{tocdepth}{4}
\pagestyle{fancy}
\fancyfoot[C]{--\ \thepage\ --}
{\makeatletter
\let\ps@jpl@in\ps@empty
\makeatother
\pagestyle{plain}
\tableofcontents
\clearpage}
\pagenumbering{arabic}
%%======================================================================================================================
%% ここから本文 ::::::::::::::::::::::::::::::::::::::::::::::::::::::::::::::::::::::::::::::::::::::::::::::::::::::::
%%======================================================================================================================

%%=============第1章=============
\chapter{まえがき}
\section{研究背景}
近年,インターネットは,情報交換・共有システムとして,社会・経済のインフラストラクチャの役割を担っている.しかし,全ての地域に高度な通信サービスを提供することは,コストが掛かる上に周波数の有限性から現実的ではない.また,ディジタルデバイド地域のインフラ構築や災害時などで本来の通信インフラの機能や性能が低下・停止した場合を考え,場所・時間を問わず利用できる情報サービスの実現が必要とされている.

そこで,従来では想定されていなかった環境を含むエンドツーエンドの情報伝達において,従来のTCP/IP(Transmission Control Protocol/Internet Protocol)技術を拡張し,中継ノード及びエンドノードの機能を再設計する必要がある.昔から車や人などが物理的に情報を運ぶやり方は存在していた.それらをTCP/IP技術の問題認識から発生した技術領域としてインターネット通信の枠組みで考えられたのが,DTNである.

TCP/IP技術におけるDTNの概念はビントン・サーフ氏によって提案され,惑星間インターネットの構築が基になっている.その惑星間インターネットには2つの大きな問題点が挙げられる.まず,惑星間というのは光速で進む電波であっても,届くのに長い時間を要する.次に,通信機が惑星によって見えなくなり,その間,通信が途絶える.これらの問題をそれぞれ,「通信遅延」,「通信途絶」として研究されてきた[2].

DTNの特徴的な技術として,蓄積運搬形転送がある.これは,再開を待つ,あるいは最適なタイミングを待つための一時的「蓄積」と,物理的に車や列車を用いて情報を運ぶ「運搬」を用いる必要がある[1].現在では,カーナビにおいてネットワークを活用した高度交通システム(ITS:Inteligent Transport Systems)の普及が急速に進んでおり,ITSにおける研究として,車々間アドック通信(VANET:Vehicular Adhoc Network)に関する研究が活発に行われている[3].そのような人や車等における蓄積運搬形転送の中で,疎密度モバイルアドホック網における感染形中継転送方式(Epidemic Routing)が確率的中継転送方式の最も代表的な方式として存在する.Epidemic Routingでは,メッセージを保持する端末が他の端末と通信可能となった際,常に保持しているメッセージの複製を他の端末に送信する.結果的に伝染病のように広がり,いずれかのメッセージが宛先端末に到達することになる.しかし,Epidemic Routingではメッセージの到達を引き換えに,端末のバッファや網内に多くの複製メッセージが残存してしまう.そのため,Recovery Scheme(回復手法)やAnti Packet(除去パケット)を使用したルーティング手法などが提案されてきた[1].

本研究では,DTN環境において,メッセージやバッファ等の資源を効率良く管理を行えば,必然的にメッセージは宛先に届けやすくなるのではないかと考えた.実験においては,「メッセージ到達率」,「メッセージ平均待ち時間」,「メッセージ削除数」に着目して,ネットワーク負荷の削減・効率的なメッセージ転送の実現を目指す.

%%=============第2章=============
\chapter{DTNにおける従来手法とその問題}
\section{DTN}
DTNとは,Delay,Disruption,Disconnection Tolerant Networkingの略で,大きな遅延や通信途絶,繋がっていない状態の方が多い環境においても,適切な情報通信を現実的なコストで実現できる技術であり,中継転送を利用することにより,エンドツーエンドの通信を可能とする[5].要素技術として,バンドルプロトコルとストアアンドフォワード方式が存在する.インターネットでのデータ転送にはトランスポート層のTCPを使用するが,通信に遅延や切断が生じると,通信品質が劣化してしまう.そのため,TCPによる通信では,エンドツーエンドのリンクが安定している必要がある.DTNでは,不安定なリンクにも対応しなければならないため,トランスポート層の上に「バンドル」と呼ばれるプロトコルを乗せることで劣通信環境に対応する.バンドルプロトコルの中でも,データ転送にはストアアンドフォワード方式を用いる[4].ストアアンドフォワード方式は,通信が不可能な状態である時にデータを蓄積し,通信が可能な状態である時にデータを転送する方式である[1].

\section{Epidemic Routing}
蓄積運搬形転送に基づく最も初期に提案された手法に,感染形中継転送方式であるEpidemic Routingと呼ばれる手法が存在する.メッセージを保持する端末が他の端末と接触をすることで通信を行い,複製メッセージを拡散させながら網内の末端までメッセージを届けることを可能にし,いずれかのメッセージが宛先端末に到達する.この手法は,網資源がある場合には,あらゆる方式の中で最も優れた遅延性能を示す[1].

しかし,問題点として,受信端末は複製メッセージをさらに複製するため,網資源を最も消費してしまう欠点がある.よって,拡散させることに重きを置いているため,効率の良いルーティングとは言い難い.Epidemic Routingのフローを図1に示す.

画像1

\section{Recovery Scheme(回復手法)}
Epidemic Routingでは,メッセージが宛先に到達した後,メッセージの複製により網内に到達したメッセージが残存してしまう可能性がある.そのためにRecovery Schemeが提案された.この手法では,メッセージが到達した宛先ノードが,到達した事を知らせるためにメッセージのIDを載せたAnti-Paketを生成する.これにより,Anti-Paketを受信したノードは,該当するメッセージを削除することで,到達したメッセージの複製を抑制する[6].Recovery Schemeのフローを図2に示す.

画像2

\section{Spray And Wait}
Spray And Waitは,メッセージ毎に複製回数が静的に設定され,メッセージを複製する度に複製回数が半分に更新される.メッセージは,Spray段階とWait段階の2つに段階を経て転送される[6].生成元ノードではある整数Lが与えられ,メッセージの複製はそれぞれForward tokenと呼ばれる整数値nを保持する.生成元ノードではn=Lとなり,転送される度にnが半減し,n=1となった時にWait段階へ移行する.Wait段階では,メッセージの複製は行われず,直接,宛先ノードに接触した時のみメッセージを転送する[7].複製回数が多すぎるとメッセージの削除が発生し,少なすぎると宛先に到達しない場合がある.

問題点として,ネットワークの大きさやノード数の変化に応じて適切な最大生成数を設定することは困難である[8].Spray And Waitのフローを図3に示す.

画像3

\section{Max Prop}
Max Propは,各ノードが保持しているメッセージに優先度を付け,他ノードと接触時に優先度の高いメッセージから順に複製メッセージを送信し,バッファの容量が不足した時に優先度の低いメッセージから順に破棄する手法である[7].Max Propのフローを図4に示す.

画像4

%%=============第3章=============
\chapter{提案手法}
\section{Priority Message Router}
Priority Message Router(以下,PMRouter)とは,端末がメッセージに優先度を設け,他ノードと接触したらそのノードのすれ違ったノード履歴より,自身の保有するメッセージの宛先と同一のメッセージを探す.存在した時に,そのメッセージの優先度を向上させ,バッファ内に存在できる時間をコントロールする.優先度の向上したメッセージに対しTTL(メッセージ有効時間)を変更することで,「メッセージ到達率」,「メッセージ平均待ち時間」,「メッセージ削除数」の改善を図る手法である.基本的なメッセージの転送は感染形中継転送方式であり,優先度を設けることから,Epidemic Routingを拡張したものとなる.以降,全てのメッセージに備わっているTTLを「ベースTTL」,優先度の向上したメッセージのTTLを「優先TTL」と呼ぶ.Priority Message Routerのフローを図5に示す.

\section{PMRouterの構成とアルゴリズム}
PMRouterでは,始めにEpidemic Routingの環境を用意する.そこで,各メッセージに優先度を設けるためのメソッドを追加する.次に各ノードのすれ違ったノードを格納するためのリストを追加し,自身の保有するメッセージの宛先と比較する.最後に,自身の保有するメッセージの宛先が他ノードのすれ違ったノードのリストに存在していれば,その宛先メッセージの優先度を向上させ,TTLをコントールする.

%%=============第4章=============
\chapter{評価}
\section{評価方法}
計算機シミュレーションにより,「メッセージ到達率」,「メッセージ平均待ち時間」,「メッセージ削除数」を評価尺度として提案方式の性能を検証する.また,従来手法(Epidemic Routing及びMax Prop)と提案手法の性能を比較し,提案手法の実用性を検証する.マップ形状は,疎密度であるdefault\_map及び過密度であるstar\_mapである.

\section{実験環境}
本研究では,The ONEシミュレータ(The Opportunistic Network Environment Simulator)を用いて実験を行う.The ONEシミュレータは,DTNの評価実験をするために作成されたフリーのネットワークシミュレータである.任意のルーティングやモビリティパターンを導入することができる.時間の遷移方式はタイムステップ型であり,標準では0.1秒毎に遷移する.また,通信可能距離と通信速度のみを考慮し,パケットのロスト,建物,電波干渉等による通信範囲の削減,保有メッセージの把握等の細かい部分は考慮していない[2].このシミュレータではバンドル層以上のみでシミュレーションを行っているため,トランスポート層以下に関わる電波の輻輳や衝突等については,パラメータで妥当な値を設定しなければならない[9].

\section{条件とシミュレーションパラメータ}
本研究を行う上での条件を以下に示す.
\begin{quote}
 \begin{itemize}
  \item 図\ref{default_map}に示すdefault\_map(初期状態のThe ONEシミュレータに搭載されている)、及び\ref{star_map}に示すstar\_mapを用いる.
  \item ノードは,歩行者と車に割り振る.
  \item 歩行者には,歩くスピードが遅い人と速い人がいることを想定し,それぞれ1.8km/hから5.4km/hとする.
  \item 車は,法定速度の多い40km/hから60km/hを想定する.
  \item 結果は,固定経路を進むシミュレーション結果を示す.
  \item 移動モデルは,各ノードが目的地を定めて最短経路で動くと仮定し,ShortestPathMapBasedMovementを用いる.
 \end{itemize}
\end{quote}
本研究を行うためのシミュレーションパラメータを表\ref{example:terminology}に示す.

\begin{table}[h]
    \begin{center}
      \caption[]{シミュレーションパラメータ}
      \label{example:terminology}
      \begin{tabular}{|c|c|}
        \hline
        移動モデル & ShortestPathMapBasedMovement\\
        \hline
        シミュレーション時間 & 43200秒\\
        \hline
        試行回数 & 1回\\
        \hline
        ノード数 & 100,300,500\\
        \hline
        TTL & 300,720 [分]\\
        \hline
        優先TTL & 各ノード数によって変更\\
        \hline
        ノードリスト
        格納数 & 20 [個]\\
        \hline
        ノードの移動速度(人) & 1.8,5.4 [km/h]\\
        \hline
        ノードの移動速度(車) & 39.96,59.94 [km/h]\\
        \hline
        発行メッセージ数 & 1460\\
        \hline
        通信速度 & 250[kbps]\\
        \hline
        通信可能距離 & 10 [m]\\
        \hline
        端末バッファ容量 & 50 [MBytes]\\
        \hline
      \end{tabular}
    \end{center}
\end{table}
%%図default_map
%%図star_map
%%=============第4章=============
\chapter{実験結果}

\section{疎密度マップ TTL=300}
TTL300における、ノード数を100、300、500に設定したときの、「メッセージ到達率」,「メッセージ平均遅延時間」,「メッセージ削除数」の結果を、表\ref{somituTtl300deliveryprob}、表\ref{somituTtl300latencyavg}、表\ref{somituTtl300dropped}にそれぞれ示す。
\begin{table}[H]
 \begin{center}
      \caption[]{メッセージ到達率(疎密度マップ TTL300)}
      \label{somituTtl300deliveryprob}
      \begin{tabular}{|c|c|c|c|}
\hline
&100&200&300\\
\hline
EpidemicRouting&0.3523&0.5326&0.6001\\
\hline
MaxPropRouting&0.4695&0.8003&0.8934\\
\hline
PriorityMessageRouting&0.3948&0.5347&0.6282\\
\hline
      \end{tabular}
    \end{center}
\end{table}

\begin{table}[H]
 \begin{center}
      \caption[]{メッセージ平均遅延時間(疎密度マップ TTL300)}
      \label{somituTtl300latencyavg}
      \begin{tabular}{|c|c|c|c|}
\hline
&100&300&500\\
\hline
EpidemicRouting&9063.5916&9473.171&8910.7454\\
\hline
MaxPropRouting&8808.0453&6908.4267&4511.4366\\
\hline
PriorityMessageRouting&11181.6049&6908.427&7345.0537\\
\hline
      \end{tabular}
    \end{center}
\end{table}

\begin{table}[H]
 \begin{center}
      \caption[]{メッセージ削除数(疎密度マップ TTL300)}
      \label{somituTtl300dropped}
      \begin{tabular}{|c|c|c|c|}
\hline
&100&200&300\\
\hline
EpidemicRouting&9562&&92280\\
\hline
PriorityMessageRouting&8722&40988&93206\\
\hline
MaxPropRouting&685,re5851&436,re38112&re86815\\
\hline
PriorityMessageRouting&8722&40988&93206\\\hline
      \end{tabular}
    \end{center}
\end{table}
\subsection{メッセージ到達率}
図\ref{ttl300deliveryprob}に示すように、データ到達率は,適切な優先TTLを設定した上で,ノード数が多くなるとEpidemic Routingとの差が大きくなった.ベースとなるTTLが300の場合,Epidemic Roitingより僅かに良い結果が得られた.一方で、MaxPropと比較すると、データ到達率は下回る結果となった。
%%図ttl300deliveryprob
\subsection{メッセージ平均遅延時間}
図\ref{ttl300latencyavg}に示すように、メッセージ平均遅延時間は,ノード数が100から,300,500と多くなるにつれて減少し、EpidemicRoutingと比較した場合においても良い結果が得られた。一方で、MaxPropと比較すると、データ到達率は下回る結果となった。
\subsection{メッセージ削除数}
図\ref{ttl300dropped}に示すように、ノード数が上がるごとにメッセージ削除数が上がり、EpidemicRoutingよりもわずかに高い削除数となった、ここでメッセージの削除数とは、TTLによるタイムアウトやバッファ溢れなどによるメッセージの削除となっており、PMrouterはTTLの制御によってメッセージの削除を行っている。また、MaxPropはバッファ溢れが起きる際、TTLに関わらず、優先度の低いメッセージを削除し、新しいメッセージを取得するという手法によって、バッファ溢れを防いでいるため、ここでのメッセージ削除数は限りなく小さく、代わりに設定条件下での意図的なメッセージ削除が行われている。
%%図ttl300dropped

\section{疎密度マップ TTL=720}
TTLを300における、ノード数を100、300、500に設定したときの、「メッセージ到達率」,「メッセージ平均遅延時間」,「メッセージ削除数」の結果を、表\ref{somituTtl720deliveryprob}、表\ref{somituTtl720latencyavg}、表\ref{somituTtl720dropped}にそれぞれ示す。
\begin{table}[H]
 \begin{center}
      \caption[]{メッセージ到達率(疎密度マップ TTL720)}
      \label{somituTtl720deliveryprob}
      \begin{tabular}{|c|c|c|c|}
\hline
&100&200&300\\
\hline
EpidemicRouting&0.3729&0.4036&0.4204\\
\hline
MaxPropRouting&0.5497&0.8058&0.8927\\
\hline
PriorityMessageRouting&0.3948&0.5347&0.6377\\
\hline
      \end{tabular}
    \end{center}
\end{table}

\begin{table}[H]
 \begin{center}
      \caption[]{メッセージ平均遅延時間(疎密度マップ TTL720)}
      \label{somituTtl720latencyavg}
      \begin{tabular}{|c|c|c|c|}
\hline
&100&300&500\\
\hline
EpidemicRouting&12670.4281&10499.5253&9447.9813\\
\hline
MaxPropRouting&11475.7576&6991.6428&9380.4749\\
\hline
PriorityMessageRouting&11181.6049&9160.6067&8219.5436\\
\hline
      \end{tabular}
    \end{center}
\end{table}

\begin{table}[H]
 \begin{center}
      \caption[]{メッセージ削除数(疎密度マップ TTL720)}
      \label{somituTtl720dropped}
      \begin{tabular}{|c|c|c|c|}
\hline
&100&200&300\\
\hline
EpidemicRouting&5265&33018&83972\\
\hline
MaxPropRouting&70,re7767&4,re38259&re86832\\
\hline
PriorityMessageRouting&8722&39950&92100\\
\hline
      \end{tabular}
    \end{center}
\end{table}
\subsection{メッセージ到達率}
図\ref{ttl300deliveryprob}と下に示す図{ttl720deliveryprob}から分かるように、データ到達率は,ベースTTLが300の時に比べ,ノード数が300の時に1\%弱上昇した.また、EpidemicRoutingはTTLが720に増加すると、ノード数が上昇するにつれて、到達率の低下が確認できたが、PMrouterではノード数の上昇による到達率の低下は確認できなかった。

%%図ttl300deliveryprob

\subsection{メッセージ平均遅延時間}
図\ref{ttl300latencyavg}と下に示す図{ttl720latencyavg}から分かるように、メッセージ平均遅延時間は,3つのルーティング手法それぞれにおいて、TTL300の時と比べて、遅延時間は長くなる結果となった。一方で、ノード数が300の場合において、PMrouterとMaxPropを比較すると、遅延時間においてMaxPropよりも優れた結果を得られた。

%%図ttl720latencyavg

\subsection{メッセージ削除数}
図\ref{ttl300latencyavg}と下に示す図{ttl720latencyavg}から分かるように、TTL増加によるメッセージ削除数は大きく変わらなかった。一方で、EpidemicRoutingと比較すると、TTL300の時と同様に、PMrouterの方が多く削除されてしまっている。

%%図{ttl720latencyavg}



\section{過密度マップ TTL=300}
TTL300における、ノード数を100、300、500に設定したときの、「メッセージ到達率」,「メッセージ平均遅延時間」,「メッセージ削除数」の結果を、表\ref{kamituTtl300deliveryprob}、表\ref{kamituTtl300latencyavg}、表\ref{kamituTtl300dropped}にそれぞれ示す。
\begin{table}[H]
 \begin{center}
      \caption[]{メッセージ到達率(過密度マップ TTL300)}
      \label{kamituTtl300deliveryprob}
      \begin{tabular}{|c|c|c|c|}
\hline
&100&200&300\\
\hline
EpidemicRouting&&&\\
\hline
MaxPropRouting&&&\\
\hline
PriorityMessageRouting&&&\\
\hline
      \end{tabular}
    \end{center}
\end{table}

\begin{table}[H]
 \begin{center}
      \caption[]{メッセージ平均遅延時間(過密度マップ TTL300)}
      \label{kamituTtl300latencyavg}
      \begin{tabular}{|c|c|c|c|}
\hline
&100&300&500\\
\hline
EpidemicRouting&&&\\
\hline
MaxPropRouting&&&\\
\hline
PriorityMessageRouting&&&\\
\hline
      \end{tabular}
    \end{center}
\end{table}

\begin{table}[H]
 \begin{center}
      \caption[]{メッセージ削除数(過密度マップ TTL300)}
      \label{kamituTtl300dropped}
      \begin{tabular}{|c|c|c|c|}
\hline
&100&300&500\\
\hline
EpidemicRouting&&&\\
\hline
MaxPropRouting&&&\\
\hline
PriorityMessageRouting&&&\\
\hline
      \end{tabular}
    \end{center}
\end{table}
\subsection{メッセージ到達率}
データ到達率は,優先度の高くなる優先TTLを設定した上で,ノード数が多くなるとEpidemic Routingとの差が大きくなった.ベースとなるTTLが300の場合,Epidemic Roitingより僅かに良い結果が得られた.
\subsection{メッセージ平均遅延時間}
メッセージ平均遅延時間は,ノード数が100から,300,500と多くなるにつれて減少した.
\subsection{メッセージ削除数}


\section{過密度マップ TTL=720}
TTL300における、ノード数を100、300、500に設定したときの、「メッセージ到達率」,「メッセージ平均遅延時間」,「メッセージ削除数」の結果を、表\ref{kamituTtl720deliveryprob}、表\ref{kamituTtl720latencyavg}、表\ref{kamituTtl720dropped}にそれぞれ示す。
\begin{table}[H]
 \begin{center}
      \caption[]{メッセージ到達率(過密度マップ TTL720)}
      \label{kamituTtl720deliveryprob}
      \begin{tabular}{|c|c|c|c|}
\hline
&100&300&500\\
\hline
EpidemicRouting&&&\\
\hline
MaxPropRouting&&&\\
\hline
PriorityMessageRouting&&&\\
\hline
      \end{tabular}
    \end{center}
\end{table}

\begin{table}[H]
 \begin{center}
      \caption[]{メッセージ平均遅延時間(過密度マップ TTL720)}
      \label{kamituTtl720latencyavg}
      \begin{tabular}{|c|c|c|c|}
\hline
&100&300&500\\
\hline
EpidemicRouting&&&\\
\hline
MaxPropRouting&&&\\
\hline
PriorityMessageRouting&&&\\
\hline
      \end{tabular}
    \end{center}
\end{table}

\begin{table}[H]
 \begin{center}
      \caption[]{メッセージ削除数(過密度マップ TTL720)}
      \label{kamituTtl720dropped}
      \begin{tabular}{|c|c|c|c|}
\hline
&100&300&500\\
\hline
EpidemicRouting&&&\\
\hline
MaxPropRouting&&&\\
\hline
PriorityMessageRouting&&&\\
\hline
      \end{tabular}
    \end{center}
\end{table}
\subsection{データ到達率}
データ到達率は,ベースTTLが300の時に比べ,ノード数が300の時に1\%弱上昇した.
\subsection{メッセージ平均遅延時間}
メッセージ平均遅延時間は,データ到達率と同様にノード数が300の時に1000秒伸びた.
\subsection{メッセージ削除数}
メッセージ削除数もまた同様に,ノード数が300の時に1000個近くのメッセージが削除された.

\chapter{結論}

\chapter{考察}


%%======================================================================================================================
%% ここまで本文 ::::::::::::::::::::::::::::::::::::::::::::::::::::::::::::::::::::::::::::::::::::::::::::::::::::::::
%%======================================================================================================================

%% 謝辞
\clearpage
\fancyhead[L]{}\fancyhead[R]{}
\renewcommand{\headrulewidth}{0truemm}
\section*{謝辞}
本研究を遂行し,卒業論文をまとめるにあたって,ご指導ならびにご助言して頂き,研究室活動全般にわたってお世話になりました菅原真司教授に感謝致します.

%%----------------------------------------------------------------------------------------------------------------------
%% 参考文献
\clearpage
\nocite{*}
\bibliographystyle{icsthesis}
\fancyhead[L]{\nouppercase{\small\leftmark}}\fancyhead[R]{}
\renewcommand{\headrulewidth}{0.3truemm}
\bibliography{thesis}

%%----------------------------------------------------------------------------------------------------------------------
%% 付録: 不要なら,最後の \end{document} を残して,これ以降の行を消す.
%%
\clearpage
\fancyhead[L]{\nouppercase{\small\leftmark}}
\fancyhead[R]{\nouppercase{\small\rightmark}}
\fancyfoot[C]{--\ \thepage\ --}
\renewcommand{\headrulewidth}{0.3truemm}
\appendix
\chapter{マップについて}
今回の実験では、シミュレータに搭載されていたヘルシンキのマップに加えて、自作の完全グラフのマップでの実験を行った.研究開始当初は参考文献に倣い、他の実在するマップでの実験を行う予定であったが、マップ導入時の問題を解決することができなかったため、断念した.\\\\
ここでは、マップの導入に関して行った手順について記す.
%ここは「付録章」である.\textsf{\yen appendix}以降に,\textsf{\yen chapter\{見出し文字列\}}と書けば,
%「付録章見出し」が設定される.
%「章見出し」と同じ文字サイズ,同じ上下余白である.

\section{osmファイルからwktファイルへの変換}



\subsection{osmファイル}
%ここは,「付録小小節」である.\textsf{\yen appendix}以降に\textsf{\yen subsubsection\{見出し文字列\}}と書けば,
%「付録小小節見出し」が設定される.
%「小小節見出し」と同じ文字サイズ,同じ上下余白である.
osm ファイルとは,オープンストリートマップという自由に利用でき,共同編集機能のある世界地図を作成する共同作業プロジェクト(参考文献OSM)にてエクスポートすることができるファイル形式である.osmファイルの状態では,TheONEシミュレータで用いることができない.
\subsection{wktファイル}
wktファイルとは、地理情報をテキスト形式で表現するファイル形式である.TheONEシミュレータで読み込めるマップの形式はwktファイルのみであるため、osmファイルは何らかの形でwktファイルに変更する必要がある.
\section{osmファイルとwktファイル}

\subsection{OpenJUMPを用いたosmファイルからwktファイルへの変換}
参考文献に倣いOpenJUMPというソフトを用いてosmファイルからwktファイルへの変換を行う.OpenJUMP とは,ユーザが地理データを表示したり編集したりすることが出るGIS(地理情報システム)である.osmファイルをOpenJUMPで開くと図\ref{}のようになり,ここから様々なノードなどの編集ができる.しかし、この方法は動作環境によってはOpenJUMP上でosmファイルを開くことができないという現象が確認できた.
\subsection{QGISを用いたosmファイルからwktファイルへの変換}
上記の問題を解決するため、QGISという別のソフトを用いる。QGIS とはGNU General Public License で提供されている,OpenJUMP と同じGIS(地理情報システム) であり,OpenJUMPと比べて多くの機能が備えられている.osmファイルに対応しており,このソフトを用いてosmファイルをshpファイルに変換する.osmファイルを別のファイル形式に変換後,そのファイルを改めてOpenJUMP上で開くことで,wktファイルへの変換を行った.
\section{マップ導入時の問題}
上記の二つのどちらの場合でも,OpenJUMPにファイルを読み込んだ際,マップの縮尺が極端に小さくなってしまった.TheONEシミュレータに導入した際も,極端に小さくなった状態で実行されてしまうため,正しい結果が得られなくなってしまった.この問題を解決することができなかったため,実在するマップでの実験を断念した.


%%=============プログラム=============
\section{プログラム}
\lstset{
    frame=single,
    numbers=left,
    tabsize=2
}
\subsection{PriorityMessageRouter.java}
\begin{framed}
\begin{verbatim}
package routing;

import core.Connection;
import core.DTNHost;
import core.Message;
import core.Settings;

import gui.EventLogPanel;

import java.util.ArrayList;
import java.util.Collection;
import java.util.LinkedList;
import java.util.List;
import java.util.Timer;
import java.util.TimerTask;

public class PriorityMessageRouter extends ActiveRouter {
    private static final int MAX_QUEUE_SIZE = 20; // キューの最大サイズ
    private LinkedList<Integer> nodeQueue; // ノード番号を格納するキュー
    private Timer timer; // タイマー

    /**
     * @param s
     **/
    public PriorityMessageRouter(Settings s) {
        super(s);
        nodeQueue = new LinkedList<>();
        timer = new Timer();
        startQueueCleanupTask();
    }

    public LinkedList<Integer> getNodeQueue() {
        return this.nodeQueue;
    }

    /**
     * @param r
     **/
    protected PriorityMessageRouter(PriorityMessageRouter r) {
        super(r);
        this.nodeQueue = new LinkedList<>(r.nodeQueue);
        this.timer = new Timer();
        startQueueCleanupTask();
    }

    private void startQueueCleanupTask() {//常に最新のすれ違った履歴にする
        timer.scheduleAtFixedRate(new TimerTask() {
            @Override
            public void run() {
                cleanUpQueue();
            }
        }, 0, 5 * 60 * 1000); // 5分ごとにノードリストをクリア
    }

    private void cleanUpQueue() {
        if (nodeQueue.size() > MAX_QUEUE_SIZE) {
            nodeQueue.removeFirst(); // キューの先頭(古い要素)を削除
        }
    }

    @Override
    public void update() {
        super.update();
        if (isTransferring() || !canStartTransfer()) {
            return;
        }

        if (exchangeDeliverableMessages() != null) {
            return;
        }

        this.tryAllMessagesToAllConnections();
    }

    @Override
    public PriorityMessageRouter replicate() {
        return new PriorityMessageRouter(this);
    }

    public void changedConnection(Connection con) {
        super.changedConnection(con);
        
        // 新しく接続されたノードのIDを取得
        if (con.isUp()) {
            DTNHost otherHost = con.getOtherNode(getHost());
            int otherAddress = otherHost.getAddress();
            
            System.out.println("-----");
            
            // ノードIDをキューに追加
            if (!nodeQueue.contains(otherAddress)) {
                if (nodeQueue.size() >= MAX_QUEUE_SIZE) {
                    nodeQueue.removeFirst(); // キューの先頭を削除
                }
                nodeQueue.addLast(otherAddress); // 新しいノードIDを追加
                System.out.println("着目ノード:" + getHost());
                System.out.println("追加されたノード:" + otherAddress + ", 
                追加後の" + getHost() + "のキュー: " + nodeQueue);
            }
            PriorityMessageRouter otherRouter = 
            (PriorityMessageRouter) otherHost.getRouter();
            System.out.println("相手ノード(ノードID:" + otherAddress + 
            ")のノードリスト: " + otherRouter.getNodeQueue());
      
            Collection<Message> myMessages = 
            getHost().getMessageCollection();
            
            System.out.println("保有メッセージ:" + myMessages);

            for (Message myMsg : myMessages) {
                DTNHost myDestination = myMsg.getTo();
                if (otherRouter.getNodeQueue().contains
                (myDestination.getAddress())) {
                    System.out.println("遭遇ノード:" + otherAddress + 
                    "はメッセージID:" + myMsg.getId() + "の宛先" + 
                    myDestination.getAddress() + "とすれ違ってます。 ");
                    System.out.print("メッセージ更新前: [");
                    for (Message msg : myMessages) {
                        System.out.print("ID:" + msg.getId() + "優先度:" + 
                        msg.getPriority() + " [" + msg.getTtl() + "] " + 
                        ", ");
                    }
                    System.out.println("]");

                    // 同一の宛先が見つかった場合、優先度を+1
                    // メッセージの優先度を上げる(getメソッド)
                    myMsg.setPriority(myMsg.getPriority() + 1);
                    
                    // 優先度が上がったメッセージのTTLを変更
                    // 無限に生存させる場合は「-1」にする
                    myMsg.setTtl(200);

                    System.out.print("メッセージ更新後: [");
                    for (Message msg : myMessages) {
                        System.out.print("ID:" + msg.getId() + "優先度:" + 
                        msg.getPriority() + " [" + msg.getTtl() + "] " + 
                        ", ");
                    }
                    System.out.println("]");  

                    // メッセージの優先度を元に戻す
                    (どのメッセージの優先度が更新されたかを見やすくするため)  
                    myMsg.setPriority(myMsg.getPriority() - 1);
                }
            }

            System.out.println("-----");
        }
    }
}
\end{verbatim}
\end{framed}

\subsection{セッティング(default\_map及びstar\_map)}
\begin{framed}
\begin{verbatim}
#
# Default settings for the simulation
#

## Scenario settings
Scenario.name = default_scenario
Scenario.simulateConnections = true
Scenario.updateInterval = 0.1
# 43200s == 12h
Scenario.endTime = 43200


## Interface-specific settings:
# type : which interface class the interface belongs to
# For different types, the sub-parameters are interface-specific
# For SimpleBroadcastInterface, the parameters are:
# transmitSpeed : transmit speed of the interface (bytes per second) 
# transmitRange : range of the interface (meters)

# "Bluetooth" interface for all nodes
btInterface.type = SimpleBroadcastInterface
# Transmit speed of 2 Mbps = 250kBps
btInterface.transmitSpeed = 250k
btInterface.transmitRange = 10



# Define 6 different node groups
Scenario.nrofHostGroups = 2 #歩行者と車の2種類

## Group-specific settings:
# groupID : Group's identifier. Used as the prefix of host names
# nrofHosts: number of hosts in the group
# movementModel: movement model of the hosts 
(valid class name from movement package)
# waitTime: minimum and maximum wait times 
(seconds) after reaching destination
# speed: minimum and maximum speeds (m/s) when moving on a path
# bufferSize: size of the message buffer (bytes)
# router: router used to route messages 
(valid class name from routing package)
# activeTimes: Time intervals when the nodes in the group are active 
(start1, end1, start2, end2, ...)
# msgTtl : TTL (minutes) of the messages created by this host group, 
default=infinite

## Group and movement model specific settings
# pois: Points Of Interest indexes and probabilities 
(poiIndex1, poiProb1, poiIndex2, poiProb2, ... )
#       for ShortestPathMapBasedMovement
# okMaps : which map nodes are OK for the group (map file indexes), 
default=all 
#          for all MapBasedMovent models
# routeFile: route's file path - for MapRouteMovement
# routeType: route's type - for MapRouteMovement


# Common settings for all groups
Group.movementModel = ShortestPathMapBasedMovement
Group.router = EpidemicRouter
Group.bufferSize = 50M
Group.waitTime = 0, 120
# All nodes have the bluetooth interface
Group.nrofInterfaces = 1
Group.interface1 = btInterface
# Walking speeds
Group.speed = 0.5, 1.5 #歩行者の速さ
# Message TTL of 300 minutes (5 hours)
Group.msgTtl = 300 #TTLによって変更

Group.nrofHosts = 50 #ノード数によって変更

# group1 (pedestrians) specific settings
Group1.groupID = p

# group2 specific settings
Group2.groupID = c
# cars can drive only on roads
Group2.okMaps = 1
# 10-50 km/h
Group2.speed = 11.1, 16.65 #車の速さ



## Message creation parameters 
# How many event generators
Events.nrof = 1
# Class of the first event generator
Events1.class = MessageEventGenerator
# (following settings are specific for the MessageEventGenerator class)
# Creation interval in seconds (one new message every 25 to 35 seconds)
Events1.interval = 25,35
# Message sizes (500kB - 1MB)
Events1.size = 500k,1M
# range of message source/destination addresses
Events1.hosts = 0,99 #ノード数によって変更
# Message ID prefix
Events1.prefix = M


## Movement model settings
# seed for movement models' pseudo random number generator (default = 0)
MovementModel.rngSeed = 1
# World's size for Movement Models without implicit size 
(width, height; meters)
MovementModel.worldSize = 4500, 3400
# How long time to move hosts in the world before real simulation
MovementModel.warmup = 1000

## Map based movement -movement model specific settings
MapBasedMovement.nrofMapFiles = 4 #star_mapのみの形成時には1にする

MapBasedMovement.mapFile1 = data/roads.wkt #star_map時には該当するwktファイルを挿入
MapBasedMovement.mapFile2 = data/main_roads.wkt #必要に応じて消す
MapBasedMovement.mapFile3 = data/pedestrian_paths.wkt #必要に応じて消す
MapBasedMovement.mapFile4 = data/shops.wkt #必要に応じて消す

## Reports - all report names have to be valid report classes

# how many reports to load
Report.nrofReports = 1
# length of the warm up period (simulated seconds)
Report.warmup = 0
# default directory of reports 
(can be overridden per Report with output setting)
Report.reportDir = reports/
# Report classes to load
Report.report1 = MessageStatsReport

## Default settings for some routers settings
ProphetRouter.secondsInTimeUnit = 30
SprayAndWaitRouter.nrofCopies = 6
SprayAndWaitRouter.binaryMode = true

## Optimization settings -- these affect the speed of the simulation
## see World class for details.
Optimization.cellSizeMult = 5
Optimization.randomizeUpdateOrder = true


## GUI settings

# GUI underlay image settings
GUI.UnderlayImage.fileName = data/helsinki_underlay.png
# Image offset in pixels (x, y)
GUI.UnderlayImage.offset = 64, 20
# Scaling factor for the image
GUI.UnderlayImage.scale = 4.75
# Image rotation (radians)
GUI.UnderlayImage.rotate = -0.015

# how many events to show in the log panel (default = 30)
GUI.EventLogPanel.nrofEvents = 100
# Regular Expression log filter 
(see Pattern-class from the Java API for RE-matching details)
#GUI.EventLogPanel.REfilter = .*p[1-9]<->p[1-9]$
\end{verbatim}
\end{framed}
\end{document}
